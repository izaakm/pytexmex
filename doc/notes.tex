\documentclass[10pt]{article}

\usepackage{amsmath,amssymb}

\begin{document}

\title{the math of texmex}
\author{Scott Olesen}

\maketitle

\section{The probability mass function}
The Poisson-lognormal pmf is a convolution of the Poisson pmf with the lognormal pdf:
\begin{align}
    \mathrm{Poilog}(n; \mu, \sigma) &= \int_0^\infty \mathrm{Poisson}(n; \lambda) \times \mathrm{Lognormal}(\lambda; \mu, \sigma) \,\mathrm{d}\lambda \\
    &= \int_0^\infty \frac{\lambda^n e^{-\lambda}}{n!} \times \frac{1}{\lambda \sigma \sqrt{2 \pi}} \exp \left\{ -\frac{(\ln \lambda - \mu)^2}{2 \sigma^2} \right\} \,\mathrm{d}\lambda \\
    &= \frac{1}{n! \, \sigma \sqrt{2\pi}} \int_{-\infty}^\infty \exp \left\{ nx - e^x - \frac{1}{2} \left(\frac{x - \mu}{\sigma}\right)^2 \right\} \,\mathrm{d}x,
\end{align}
where $x = \ln \lambda$.

This integral presents a few challenges:
\begin{itemize}
\item When $n$ is large, $n!$ becomes difficult to compute.
\item When $x$ is large, the double exponential leads to overflows.
\end{itemize}
I got around these problems by:
\begin{itemize}
\item computing $\log n!$, i.e., $\log \Gamma(n + 1)$, which tends to be tractable, and put that
inside the integrand, and
\item being smart about the bounds of the integral.
\end{itemize}
Specifically, I examine the three terms inside the integral:
\begin{equation}
    f(x) \equiv nx - e^x - \frac{1}{2} \left(\frac{x - \mu}{\sigma}\right)^2
\end{equation}
It is easy to numerically compute the $x^\star$ that maximizes $f$. To find the lower bound, I establish
some threshold $T$ (say, $10^{-10}$) and iterate:
\begin{enumerate}
\item Guess a difference $\Delta x$, say $1.0$.
\item If $f(x^\star - \Delta x) / f(x^\star) < T$, then the lower bound $x^\star - \Delta x$ is far away
enough from $x^\star$ that integrating down to that point will capture most of the value of the integral.
\item If the ratio does not reach down to the threshold $T$, multiply $\Delta x$ by some factor, say $2.0$,
and try again.
\end{enumerate}
To find the upper bound, repeat but with $x^\star + \Delta x$.

\section{Maximum likelihood estimation}
One trick here is that $\sigma$ must be nonnegative. The easy way to establish this
constraint is to perform the optimization over $(\mu, \log \sigma)$, which is unconstrained.

The other important trick is picking the initial conditions. Good guesses for $\mu$ and
$\sigma$ are, I think, just the mean and standard deviation of the logarithms of the data
themselves, i.e., the maximum likelihood parameters for a plain lognormal distribution.
This works most of the time, but I've found that, to cover all cases, I've needed to do some
mild grid searching. For all the ocean and mouse gut microbiome samples I've used, I've been
able to find a match using $\sigma = 1.0$ and $\mu \in \{-2.0, -1.0, 0, 1.0, 2.0\}$.

\end{document}
